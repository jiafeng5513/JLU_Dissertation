% Copyright (c) 2008-2009 solAvethis
% Copyright (c) 2010-2016,2018 Casper Ti. Vector
% Copyright (c) 2020-2021 jiafeng5513
% Public domain.Aa


% 采用了自定义的(包括大小写不同于原文件的)字体文件名,
% 并改动 ctex.cfg 等配置文件的用户请自行加入 nofonts 选项;
% 其它用户不用加入 nofonts 选项,加入之后反而会产生错误。
% 默认为书籍排版即双面打印,章节从偶数页开始
% oneside:单面打印
% openany:章节从任意页码开始
\documentclass[UTF8,openany,oneside]{jlutex}
% 如果的确须要使脚注按页编号的话,可以去掉后面 footmisc 包的注释。
% 注意:在启用此设定的情况下,可能要多编译一次以产生正确的脚注编号。
\usepackage[perpage]{footmisc}

\usepackage{CJK}

\usepackage{booktabs}
\usepackage{algorithm}
\usepackage{algorithmicx}  
\usepackage{appendix}
\usepackage{algpseudocode}
\usepackage{setspace}

\floatname{algorithm}{算法}
\renewcommand{\algorithmicrequire}{\textbf{输入}}
\renewcommand{\algorithmicensure}{\textbf{输出}}
\newcommand*{\rom}[1]{\expandafter (\romannumeral #1)}

% 使用 biblatex 排版参考文献,并规定其格式(详见 biblatex-caspervector 的文档)。
% 这里按照西文文献在前,中文文献在后排序(“sorting = ecnyt”);
% 若须按照中文文献在前,西文文献在后排序,请设置“sorting = cenyt”;
% 若须按照引用顺序排序,请设置“sorting = none”。
% 若须在排序中实现更复杂的需求,请参考 biblatex-caspervector 的文档。
\usepackage[backend = biber, style = caspervector, utf8]{biblatex}
%\usepackage{subfigure}
% 对于 linespread 值的计算过程有兴趣的同学可以参考 jlutex.cls。
\renewcommand*{\bibfont}{\zihao{5}\linespread{1.27}\selectfont}
% 按学校要求设定参考文献列表的段间距。
\setlength{\bibitemsep}{3bp}

% 设定文档的基本信息。
\jlutexinfo{
	cnumber = {TP391}, %根据自己的专业进行修改
	UnitCode = {10183}, %请根据自己学院的模板进行相应修改
	level = {硕士},
	DegreeCategory = {学术硕士},
	securitylevel = {内部2年},
	cthesisname = {硕士学位论文},ethesisname = {Master Thesis},
	ctitle = {吉林大学学位论文LaTeX模板示例}, etitle = {Sample of LaTexAa Template for Jilin University Dissertation},
	cauthor = {}, eauthor = {},
	address = {吉林省长春市朝阳区前进大街2699号(130012)},
	telephone={},
	studentid = {},
	date = {2020年3月},
	school = {计算机科学与技术学院},
	cmajor = {计算机应用技术}, emajor = {Computer application technology},
	direction = {计算机图形学与数字媒体},
	cmentor = {}, ementor = {},
	ckeywords = {吉林大学,LaTeX,学位论文}, ekeywords = {Jilin University, LaTeX, Dissertation}
}

\addbibresource{ref.bib}	% 载入参考文献数据库(注意不要省略“.bib”)。

% 普通用户可删除此段,并相应地删除 chap/*.tex 中的
% “\jlutexffaq % 中文测试文字。”一行。
\usepackage{color}
\emergencystretch=2em %边缘距离容差

%给目录中的大章节名后面加上点号
\usepackage[subfigure]{tocloft} %模板中用了subfigure,不加此选项会产生冲突
\renewcommand{\cftchapleader}{\cftdotfill{\cftdotsep}}

%对号和错号
\usepackage{amssymb}% http://ctan.org/pkg/amssymb
\usepackage{pifont}% http://ctan.org/pkg/pifont
\newcommand{\cmark}{\ding{51}}%
\newcommand{\xmark}{\ding{55}}%
%修改全局的表格字号
\usepackage{etoolbox}
\BeforeBeginEnvironment{tabular}{\zihao{-5}}
%表格背景色
\usepackage{colortbl}
\usepackage{fancyhdr}

\pagestyle{fancy}
\fancypagestyle{toc}{
\fancyhf{}%
%\fancyfoot{thepage}
\renewcommand{\headrulewidth}{0pt}%
}

\begin{document}{}
	% 以下为正文之前的部分,默认不进行章节编号。
	\frontmatter
	\setcounter{page}{0}		% 重置页码计数器,用大写罗马数字排版此部分页码。\\
	\pagestyle{empty}% 此后到下一 \pagestyle 命令之前不排版页眉或页脚。
	\maketitle					% 生成封面。
	\innertitle
	\setlength{\baselineskip}{30pt}
{
% 此处不用 \specialchap,因为学校要求目录不包括其自己及其之前的内容。
\centerline{\songti\Large吉林大学硕士学位论文原创性声明}
\vskip 1.5cm
% 综合学校的书面要求及 Word 模版来看,版权声明页不用加页眉、页脚。
\thispagestyle{empty}

\songti{本人郑重声明:所呈交学位论文,是本人在指导教师的指导下,独立进行研究工作所取得的成果。
	除文中已经注明引用的内容外,本论文不包含任何其他个人或集体已经发表或撰写过的作品成果。
	对本文的研究做出重要贡献的个人和集体,均已在文中以明确方式标明。
	本人完全意识到本声明的法律结果由本人承担。}

\vskip 5.5cm

\songti{\normalsize
	\hspace{7.0cm}学位论文作者签名:
	
	\hspace{6.8cm} 日期:\hspace{2.0cm}年\hspace{1.0cm}月\hspace{1.0cm}日
}
}


	% 原创性声明
	%\cleardoublepage
	\begin{mcopyright}
	\centerline{\Large 关于学位论文使用授权的声明}
	\thispagestyle{empty}
	{
		\vskip 1.5cm
		本人完全了解吉林大学有关保留、使用学位论文的规定,同意吉林大学保留或向国家有关部门或机构送交论文的复印件和电子版,允许论文被查阅和借阅;本人授权吉林大学可以将本学位论文的全部或部分内容编入有关数据库进行检索,可以采用影印、缩印或其他复制手段保存论文和汇编本学位论文。
		
		(保密论文在解密后应遵守此规定)\\
	}
\end{mcopyright}	% 投稿声明
	%\cleardoublepage
	
	\pagenumbering{Roman}
	\setcounter{page}{1}		% 重置页码计数器,用大写罗马数字排版此部分页码。\\
	\pagestyle{myheading}	% 此后到下一 \pagestyle 命令之前正常排版页眉和页脚。
	\ctexset{chapter={pagestyle=myheading}}
	\begin{cabstract}

在此处输入中文摘要正文。

\end{cabstract}

\begin{eabstract}

Put English abstract text here.

\end{eabstract}		% 中西文摘要
	\tableofcontents
	\thispagestyle{myheading}
	\ctexset{chapter={pagestyle=plain}}

	\pagestyle{plain}	% 此后到下一 \pagestyle 命令之前正常排版页眉和页脚。
	\mainmatter% 以下为正文部分,默认要进行章节编号。

	\chapter{引言}
\jlutexffaq % 中文测试文字。
\\
引用中文参考文献\supercite {test-zh},引用英文参考文献\cite{test-en}。
% vim:ts=4:sw=4
  % 引言
	\chapter{章节}
\jlutexffaq % 中文测试文字。

% vim:ts=4:sw=4
  % 相关工作 
	\chapter{工作主体内容}
\section{插入图片}
如图\ref{fig:samplefigure}所示,请在Figure文件夹中存储论文中的图片。
同时,在图片中出现的文字请尽量使用黑体以及Arial字体以保持和图题相同。
% TODO: \usepackage{graphicx} required
\begin{figure}[htb]
	\centering
	\includegraphics[width=1.0\linewidth]{Figure/SampleFigure}
	\caption{示例图片Sample Figure}
	\label{fig:samplefigure}
\end{figure}

\section{插入公式}
如公式\ref{con:RayDef}所示,模板会自动为公式标号且应用学校要求的格式。
\begin{equation}
	L = Origin + HitT \cdot Direction
	\label{con:RayDef}
\end{equation}  % 本文方法
	\chapter{实验和结果}
\section{插入表格}
大多数论文需要使用表格来记录实验数据,根据学校要求,如表~\ref{tab:RTCon}所示,需要使用三线式表格以展示表格内容。
\begin{table}[htb]{
		\centering
		\caption{文档处理实验室主要人员成绩单}
		\label{tab:RTCon}
		\setlength{\tabcolsep}{8mm}
		\begin{tabular}{lcccc}
			\toprule
			姓名 & 语文 & 数学 & 外语 & 政治 \\ 
			\midrule
			安娜 & 100 & 100 & 100 & 100 \\ 
			小羽  & 99 & 95 & 90 & 95 \\ 
			攀攀  & 98  &98 & 94 &99 \\ 
			阿福  & 97& 90 & 90 & 97 \\ 
			大姐头 & 97 &99 & 100 &100 \\ 
			小付 & 97 & 99& 100 & 98\\  
			\midrule
			\textbf{平均} &\textbf{100} & \textbf{100}  & \textbf{100}  & \textbf{100}  \\ 
			\bottomrule
		\end{tabular}\\
	}
\end{table}  % 实验
	\chapter{交叉引用和参考文献}
  % 结论与展望
	% 正文中的附录部分。	
	% 排版参考文献列表。bibintoc 选项使“参考文献”出现在目录中;
	% 如果同时要使参考文献列表参与章节编号,可将“bibintoc”改为“bibnumbered”。
	\normalem
	\printbibliography[heading = bibintoc]
	
	% 各附录。
	\appendix
	\chapter{附录}
\section{TODO}
目前由Tex Live2020生成的文件存在页眉中章节号变为英文的问题。
推荐在安装Tex Live时取消选择自动更新。
我们致力于解决此问题,欢迎各位同学沟通解决方案。
此问题预计将会在下一个版本中修复。
% vim:ts=4:sw=4

	% 以下为正文之后的部分,默认不进行章节编号。
	% heading=bibintoc    : 将参考文献加入目录中
	% heading=bibnumbered : 参考文献列表参与章节
	 % 作者简介以及科研成果
	\backmatter					                    
	\chapter{作者简介及科研成果}
\section*{作者简介}
此处留白,不需要填写任何内容。

\section*{科研成果}
\vskip 10pt
\begin{itemize}[itemsep= 7pt, labelsep= 10pt, leftmargin = 25pt, topsep = 10pt]
	\small
	\setlength{\baselineskip}{16pt}
	\item[{$\left[1\right]$}]
	除导师外第一作者. \textit{"Paper name"} [C]. In: \textit{Conference name}, \textbf{Year}, pages-pages.  doi: \texttt{doi}. (EI Accession number: \texttt{accession number}).
	
	\item[{$\left[2\right]$}]
	除导师外第一作者. "\textit{Paper name}" [J]. \textit{Journal name}, \textbf{Year}. doi: \texttt{doi}. (CCF C类期刊).
	
	\item[{$\left[3\right]$}]
	基于***的***系统V1.0. 中国. 计算机软件著作权. 登记号:*********.
\end{itemize}
	%\chapter{致谢}
感谢安娜学姐制作Latex模板。
感谢P\_Lee在对于修改提出的各种建议以及帮助。
感谢小付提出的论文撰写建议。
感谢Ris对于论文模板内容的更新。

同时,感谢各位参与测试的同学。也希望各位同学可以积极参与到模板的测试和修改中来。
% vim:ts=4:sw=4
	% 致谢

\end{document}

% vim:ts=4:sw=4
